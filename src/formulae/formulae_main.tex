\documentclass[12pt,letterpaper,oneside,reqno]{amsart}
\usepackage{amsfonts}
\usepackage{amsmath}
\usepackage{amssymb}
\usepackage{amsthm}
\usepackage{float}
\usepackage{mathrsfs}
\usepackage{colonequals}
\usepackage[font=small,labelfont=bf]{caption}
\usepackage[left=1in,right=1in,bottom=1in,top=1in]{geometry}
\usepackage[pdfpagelabels,hyperindex,colorlinks=true,linkcolor=blue,urlcolor=magenta,citecolor=green]{hyperref}
\usepackage{graphicx}
\linespread{1.7}
\emergencystretch=1em
\usepackage{array}
\usepackage{etoolbox}
\apptocmd{\sloppy}{\hbadness 10000\relax}{}{}
\raggedbottom

\newcommand \anglePower [2]{\langle #1 \rangle \sp{#2}}
\newcommand \bernoulli [2][B] {{#1}\sb{#2}}
\newcommand \curvePower [2]{\{#1\}\sp{#2}}
\newcommand \coeffA [3][A] {{\mathbf{#1}} \sb{#2,#3}}
\newcommand \polynomialP [4][P]{{\mathbf{#1}}\sp{#2} \sb{#3}(#4)}

% ordinary derivatives
\newcommand \derivative [2] {\frac{d}{d #2} #1}                              % 1 - function; 2 - variable;
\newcommand \pderivative [2] {\frac{\partial #1}{\partial #2}}               % 1 - function; 2 - variable;
\newcommand \qderivative [1] {D_{q} #1}                                      % 1 - function
\newcommand \nqderivative [1] {D_{n,q} #1}                                   % 1 - function
\newcommand \qpowerDerivative [1] {\mathcal{D}_q #1}                         % 1 - function;
\newcommand \finiteDifference [1] {\Delta #1}                                % 1 - function;
\newcommand \pTsDerivative [2] {\frac{\partial #1}{\Delta #2}}               % 1 - function; 2 - variable;

% high order derivatives
\newcommand \derivativeHO [3] {\frac{d^{#3}}{d {#2}^{#3}} #1}                % 1 - function; 2 - variable; 3 - order
\newcommand \pderivativeHO [3]{\frac{\partial^{#3}}{\partial {#2}^{#3}} #1}
\newcommand \qderivativeHO [2] {D_{q}^{#2} #1}                               % 1 - function; 2 - order
\newcommand \qpowerDerivativeHO [2] {\mathcal{D}_{q}^{#2} #1}                % 1 - function; 2 - order
\newcommand \finiteDifferenceHO [2] {\Delta^{#2} #1}                         % 1 - function; 2 - order
\newcommand \pTsDerivativeHO [3] {\frac{\partial^{#3}}{\Delta {#2}^{#3}} #1} % 1 - function; 2 - variable;

\newtheorem{thm}{Theorem}[section]
\newtheorem{cor}[thm]{Corollary}
\newtheorem{lem}[thm]{Lemma}
\newtheorem{examp}[thm]{Example}
\newtheorem{conj}[thm]{Conjecture}
\newtheorem{defn}[thm]{Definition}

\numberwithin{equation}{section}

\title[Polynomial identities involving Central Factorial numbers]
{Polynomial identities involving Central Factorial numbers}
\author[Petro Kolosov]{Petro Kolosov}
\email{kolosovp94@gmail.com}
\keywords{
    Polynomials,
    Polynomial identities,
    Faulhaber's formula,
    Cental Factorial Numbers
}
\urladdr{https://kolosovpetro.github.io}
\subjclass[2010]{26E70, 05A30}
\date{\today}
\hypersetup{
    pdftitle={Polynomial identities involving Central Factorial numbers},
    pdfsubject={
        Polynomials,
        Polynomial identities,
        Faulhaber's formula,
        Cental Factorial Numbers
    },
    pdfauthor={Petro Kolosov},
    pdfkeywords={
        Polynomials,
        Polynomial identities,
        Faulhaber's formula,
        Cental Factorial Numbers
    }
}
\begin{document}
    \begin{abstract}
    \end{abstract}

    \maketitle

    \tableofcontents

    \section{Formulae}\label{sec:formulae}
    From OEIS, note that this is not Central factorial number itself, this formula is in the mathematica package as
\texttt{OEISFormula}
\begin{equation*}
    T_{\mathrm{OEIS}} (n,k) = \frac{1}{m} \sum_{j=0}^{m} (-1)^{j} \binom{2m}{j} (m-j)^{2n}
\end{equation*}
where $m=n-k+1$.
So that
\begin{equation}
    \begin{split}
        T_{\mathrm{OEIS}} (n,k) &= \frac{1}{n-k+1} \sum_{j=0}^{n-k+1} (-1)^{j} \binom{2(n-k+1)}{j} ([n-k+1]-j)^{2n} \\
        T_{\mathrm{OEIS}} (n,k) &= \frac{1}{n-k+1} \sum_{j=0}^{n-k+1} (-1)^{j} \binom{2n-2k+2}{j} (n-k+1-j)^{2n}
    \end{split}\label{eq:luschny-oeis}
\end{equation}
Furthermore, $T_{\mathrm{OEIS}}$ may be turned into changing the summation order from $n-k+1$ to $k$
\begin{equation*}
    T_{\mathrm{OEIS}} (n, n-k) = \frac{1}{k} \sum_{j=0}^{k} (-1)^{j} \binom{2k}{j} (k-j)^{2n}
\end{equation*}
Also, OEIS sequence is defined by
\begin{equation*}
    T_{\mathrm{OEIS}} = (2(n-k) + 1)! T(2n, 2n-2k)
\end{equation*}
where $T(2n, 2n-2k)$ are central factorial numbers.
From stackoverflow, these are pure central factorial numbers already
\begin{equation*}
    k! T(n,k) = \sum_{j=0} \binom{k}{j} (-1)^{j} \left( \frac{1}{2}k - j \right)^{n}
\end{equation*}
So that central factorial number is, this is the function \texttt{Central1(n,k)} in mathematica package
and it is true and holds in mathematica program
\begin{equation*}
    T(n,k) = \frac{1}{k!} \sum_{j=0} \binom{k}{j} (-1)^{j} \left( \frac{1}{2}k - j \right)^{n}
\end{equation*}
Let be $(k-1)! T(n,k)$
\begin{equation*}
(k-1)
    !T(n,k) = \frac{1}{k} \sum_{j=0} \binom{k}{j} (-1)^{j} \left( \frac{1}{2}k - j \right)^{n}
\end{equation*}
Let be $(2k-1)! T(2n, 2k)$ in is true and checked in mathematica as \texttt{KnuthCoefficient2}
\begin{equation*}
    (2k-1)! \cdot T(2n, 2k) = \frac{1}{2k} \sum_{j=0}^{2k} \binom{2k}{j} (-1)^{j} (k-j)^{2n}
\end{equation*}
Let be $(2k-1)! T(2n, 2k)$ in is true and checked in mathematica as \texttt{KnuthCoefficient3}
\begin{equation*}
    (2k-1)! \cdot T(2n, 2k) = \frac{1}{k} \sum_{j=0}^{k} \binom{2k}{j} (-1)^{j} (k-j)^{2n}
\end{equation*}
Let be $(2k-1)! T(2n, 2k)$ in is true and checked in mathematica as \texttt{KnuthCoefficient4}
\begin{equation*}
(2k-1)! \cdot T(2n, 2k) = \frac{1}{k} \sum_{j=0}^{k} \binom{2k}{k-j} (-1)^{k-j} j^{2n}
\end{equation*}

\end{document}