\documentclass[12pt,letterpaper,oneside,reqno]{amsart}
\usepackage{amsfonts}
\usepackage{amsmath}
\usepackage{amssymb}
\usepackage{amsthm}
\usepackage{float}
\usepackage{mathrsfs}
\usepackage{colonequals}
\usepackage[font=small,labelfont=bf]{caption}
\usepackage[left=1in,right=1in,bottom=1in,top=1in]{geometry}
\usepackage[pdfpagelabels,hyperindex,colorlinks=true,linkcolor=blue,urlcolor=magenta,citecolor=green]{hyperref}
\usepackage{graphicx}
\linespread{1.7}
\emergencystretch=1em
\usepackage{array}
\usepackage{etoolbox}
\apptocmd{\sloppy}{\hbadness 10000\relax}{}{}
\raggedbottom

\newcommand \anglePower [2]{\langle #1 \rangle \sp{#2}}
\newcommand \bernoulli [2][B] {{#1}\sb{#2}}
\newcommand \curvePower [2]{\{#1\}\sp{#2}}
\newcommand \coeffA [3][A] {{\mathbf{#1}} \sb{#2,#3}}
\newcommand \polynomialP [4][P]{{\mathbf{#1}}\sp{#2} \sb{#3}(#4)}

% ordinary derivatives
\newcommand \derivative [2] {\frac{d}{d #2} #1}                              % 1 - function; 2 - variable;
\newcommand \pderivative [2] {\frac{\partial #1}{\partial #2}}               % 1 - function; 2 - variable;
\newcommand \qderivative [1] {D_{q} #1}                                      % 1 - function
\newcommand \nqderivative [1] {D_{n,q} #1}                                   % 1 - function
\newcommand \qpowerDerivative [1] {\mathcal{D}_q #1}                         % 1 - function;
\newcommand \finiteDifference [1] {\Delta #1}                                % 1 - function;
\newcommand \pTsDerivative [2] {\frac{\partial #1}{\Delta #2}}               % 1 - function; 2 - variable;

% high order derivatives
\newcommand \derivativeHO [3] {\frac{d^{#3}}{d {#2}^{#3}} #1}                % 1 - function; 2 - variable; 3 - order
\newcommand \pderivativeHO [3]{\frac{\partial^{#3}}{\partial {#2}^{#3}} #1}
\newcommand \qderivativeHO [2] {D_{q}^{#2} #1}                               % 1 - function; 2 - order
\newcommand \qpowerDerivativeHO [2] {\mathcal{D}_{q}^{#2} #1}                % 1 - function; 2 - order
\newcommand \finiteDifferenceHO [2] {\Delta^{#2} #1}                         % 1 - function; 2 - order
\newcommand \pTsDerivativeHO [3] {\frac{\partial^{#3}}{\Delta {#2}^{#3}} #1} % 1 - function; 2 - variable;

\newtheorem{thm}{Theorem}[section]
\newtheorem{cor}[thm]{Corollary}
\newtheorem{lem}[thm]{Lemma}
\newtheorem{examp}[thm]{Example}
\newtheorem{conj}[thm]{Conjecture}
\newtheorem{defn}[thm]{Definition}

\numberwithin{equation}{section}

\title[Definitions]
{Definitions}
\author[Petro Kolosov]{Petro Kolosov}
\keywords{
    Polynomials,
    Polynomial identities,
    Faulhaber's formula,
    Cental Factorial Numbers
}
\subjclass[2010]{26E70, 05A30}
\date{\today}
\hypersetup{
    pdftitle={Definitions},
    pdfsubject={
        Polynomials,
        Polynomial identities,
        Faulhaber's formula,
        Cental Factorial Numbers
    },
    pdfauthor={Petro Kolosov},
    pdfkeywords={
        Polynomials,
        Polynomial identities,
        Faulhaber's formula,
        Cental Factorial Numbers
    }
}
\begin{document}
    \begin{abstract}
        Definitions
    \end{abstract}

    \maketitle

    \tableofcontents


    \section{Definitions}\label{sec:definitions}
    \begin{itemize}
    \item $T(n,k)$ recursively defines central factorial numbers of the second kind
    (in the context of Knuth and Riordan (see references)).
    It is defined in mathematica package as \texttt{CentralFactorialNumber1}
    \begin{equation*}
        \begin{cases}
            T(n,1) &=1 \\
            T(n,n) &=1 \\
            T(n,k) &=T(n-1, k-1) + k^2 T(n-1, k)
        \end{cases}
    \end{equation*}
    \item $T(n,k)$ is central factorial number defined as \texttt{CentralFactorialNumber2} in mathematica package
    \begin{equation*}
        T(n,k) = \frac{1}{k!} \sum_{j=0}^{k} \binom{k}{j} (-1)^{j} \left( \frac{1}{2}k - j \right)^{n}
    \end{equation*}
    \item Identity in central factorial numbers defined as \texttt{CFNIdentity1} in mathematica package
    \begin{equation*}
    (2k-1)! T(2n,2k) = \frac{1}{k} \sum_{j=0}^{k} (-1)^j \binom{2k}{j} (k-j)^{2n}
    \end{equation*}
    \item Identity in central factorial numbers defined as \texttt{CFNIdentity2} in mathematica package
    \begin{equation*}
    (2k-1)! T(2n,2k) = \frac{1}{k} \sum_{j=0}^{k} (-1)^{k-j} \binom{2k}{k-j} j^{2n}
    \end{equation*}
    \item Identity in central factorial numbers defined as \texttt{CFNIdentity3} in mathematica package
    \begin{equation*}
    (2k-1)! T(2n, 2k) = \frac{1}{2k} \sum_{j=0}^{2k} (-1)^{j} \binom{2k}{j} (k-j)^{2n}
    \end{equation*}
    \item Identity in odd power polynomial
    \begin{equation*}
        n^{2m-1} = \sum_{k=1}^{m} (2k-1)! T(2m,2k) \binom{n+k-1}{2k-1}
    \end{equation*}
    It is defined as \texttt{OddPowerIdentity1} in mathematica package
    \begin{equation*}
        n^{2m-1} = \sum_{k=1}^{m} \mathtt{CFNIdentity1} (m,k) \binom{n+k-1}{2k-1}
    \end{equation*}
    It is defined as \texttt{OddPowerIdentity2} in mathematica package
    \begin{equation*}
        n^{2m-1} = \sum_{k=1}^{m} \mathtt{CFNIdentity2} (m,k) \binom{n+k-1}{2k-1}
    \end{equation*}
    It is defined as \texttt{OddPowerIdentity3} in mathematica package
    \begin{equation*}
        n^{2m-1} = \sum_{k=1}^{m} \mathtt{CFNIdentity3} (m,k) \binom{n+k-1}{2k-1}
    \end{equation*}
    \item Polynomial identity in odd powers defined as \texttt{OddPowerIdentity21}
    \begin{equation*}
        n^{2m-1} = \sum_{k=1}^{m} \sum_{j=0}^{k} \frac{(-1)^{k-j}}{k} \binom{n+k-1}{2k-1} \binom{2k}{k-j} j^{2m}
    \end{equation*}
    \item Polynomial identity in odd powers defined as \texttt{OddPowerIdentity22}
    \begin{equation*}
        n^{2m-1} = \sum_{k=1}^{m} \sum_{j=0}^{k} \frac{(-1)^{k-j}}{k} \frac{2k}{n+k} \binom{n+k}{2k} \binom{2k}{k-j} j^{2m}
    \end{equation*}
\end{itemize}

\end{document}