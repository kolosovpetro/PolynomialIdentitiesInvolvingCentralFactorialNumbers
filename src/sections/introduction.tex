Central factorial numbers quietly often appear in literature, like for instance, in Riordan's
Combinatorial identities~\cite[p. 217, Table 6.2(a)]{riordan1968combinatorial}
\begin{equation}
    k! T(n,k) = \sum_{j=0}^{k} \binom{k}{j} (-1)^{j} \left( \frac{1}{2}k - j \right)^{n}
    \label{eq:cfn-riordan}
\end{equation}
where $T(n,k)$ is central factorial number.
Also, the book~\cite[formula (10a)]{carlitz_riordan_1963} references central factorial numbers as
\begin{equation*}
    K_{rs} = \frac{1}{(2s)!} \sum_{t=0}^{2s} (-1)^t \binom{2s}{t} (s-t)^{2r+2}
\end{equation*}
%J.F.\ Steffenson mentions central factorial numbers in his book~\cite{steffensen1927interpolation}
%in context of central difference of polynomial $x^r$ at zero
%\begin{equation*}
%    \delta^m 0^r = \sum_{0}^{m} (-1)^v \binom{m}{v} \left( \frac{m}{2} - v \right)^r
%\end{equation*}
D. E. Knuth gives the following recurrence for the central factorial numbers~\cite[p. 284]{knuth1993johann}
\begin{equation*}
    T(2m+2, 2k) = k^2 T(2m,2k) + T(2m, 2k - 2)
\end{equation*}
In The On-Line Encyclopedia of Integer Sequences, central factorial numbers~\cite{sloane2000centralfactorial}
appear to be defined via the following recurrence, let be $U(n,k) = T(2n,2k)$ then
\begin{equation*}
    U(n,k) = \begin{cases}
        1, & \text{if } k=1; \\
        1, & \text{if } k=n; \\
        U(n-1, k-1) + k^2 U(n-1, k), & \text{otherwise}
    \end{cases}
\end{equation*}
It is important to note that central factorial numbers are closely related to the central difference operator $\delta$,
Newton interpolation formula and central factorials $\centralFactorial{x}{n}$ and could be derived respectively.
The derivation of central factorial numbers by means of central difference operator $\delta$,
Newton interpolation formula and central factorials $\centralFactorial{x}{n}$
is shown at~\cite{scheuer2020mathstackexchange}.
So there are few central factorial numbers identities at our disposal.
The first is
\begin{equation}
(2k-1)
    !T(2n,2k) = \frac{1}{k} \sum_{j=0}^{k} (-1)^j \binom{2k}{j} (k-j)^{2n}\label{eq:cfn-identity-1}
\end{equation}
Such that it is derived from the formula given by the OEIS sequence A303675~\cite{kolosov2018coefficients}, that is
\begin{equation*}
    A303675(n,k) = \frac{1}{n-k+1} \sum_{j=0}^{n-k+1} (-1)^{j} \binom{2[n-k+1]}{j} ([n-k+1]-j)^{2n}
\end{equation*}
Note that the term $n-k+1$ is given only to serve proper sequence offset and can be replaced easily by iterator $k$
so that we have arrived to the identity~\eqref{eq:cfn-identity-1}.
The second identity is actively used in current manuscript is that
\begin{equation}
(2k-1)
    !T(2n,2k) = \frac{1}{k} \sum_{j=0}^{k} (-1)^{k-j} \binom{2k}{k-j} j^{2n}\label{eq:cfn-identity-2}
\end{equation}
and it is derived from~\eqref{eq:cfn-identity-1} by means
of symmetry of binomial coefficients $\binom{n}{k} = \binom{n}{n-k}$.
The third one and final identity we base our results on involves a partial case of the equation~\eqref{eq:cfn-riordan}
\begin{equation}
(2k-1)
    !T(2n, 2k) = \frac{1}{2k} \sum_{j=0}^{2k} \binom{2k}{j} (-1)^{j} (k-j)^{2n}\label{eq:cfn-identity-3}
\end{equation}
So that now let's stick to the results of D. E. Knuth's work~\cite{knuth1993johann},
in particular to the polynomial identities.
For odd powers we have
\begin{equation}
    n^{2m-1} = \sum_{k=1}^{m} (2k-1)! T(2m,2k) \binom{n+k-1}{2k-1}\label{eq:knuth-odd-power}
\end{equation}
And for any natural $m$ we have polynomial identity
\begin{equation}
    x^m = \sum_{k=1}^{m} T(m, k) \centralFactorial{x}{k}\label{eq:knuth-power-identity}
\end{equation}
where $\centralFactorial{x}{k}$ denotes central factorial defined by
\begin{equation*}
    \centralFactorial{x}{n} = x \fallingFactorial{x+\frac{n}{2}-1}{n-1}
\end{equation*}
where $\fallingFactorial{n}{k} = n (n-1) (n-2) \cdots (n-k+1)$ denotes falling factorial in Knuth's notation.
In particular,
\begin{equation*}
    \centralFactorial{x}{n}
    = x \left( x+\frac{n}{2}-1 \right) \left( x+\frac{n}{2}-1 \right) \cdots \left (x+\frac{n}{2}-n-1 \right)
    = x \prod_{k=1}^{n-1} \left( x+\frac{n}{2}-k \right)
\end{equation*}
So that having the whole context of the topic, we can derive and discuss few other polynomial identities smoothly.

For example, given the Knuth's odd power identity~\eqref{eq:knuth-odd-power}
and identity in central factorial numbers~\eqref{eq:cfn-identity-1}
we easily obtain
\begin{equation*}
    n^{2m-1} = \sum_{k=1}^{m} \sum_{j=0}^{k} \frac{(-1)^j}{k} \binom{2k}{j} \binom{n+k-1}{2k-1} (k-j)^{2m}
\end{equation*}
Furthermore, by means of binomial identity $\frac{k}{n} \binom{n}{k} = \binom{n-1}{k-1}$ we get
\footnote{The majority of binomial identities used in this paper are given by Gross J. L.
in his book~\cite{gross2016combinatorial}, some of the chapters available online.}
\begin{equation*}
    n^{2m-1} = \sum_{k=1}^{m} \sum_{j=0}^{k} \frac{(-1)^{j}}{k} \frac{2k}{n+k} \binom{n+k}{2k} \binom{2k}{j} (k-j)^{2m}
\end{equation*}
Collapsing common terms $k$ and applying binomial identity $\binom{n}{m} \binom{m}{k} = \binom{n}{k} \binom{n-k}{m-k}$
another polynomial identity follows
\begin{equation*}
    n^{2m-1} = 2\sum_{k=1}^{m} \sum_{j=0}^{k} \frac{(-1)^{j}}{n+k} \binom{n+k}{j} \binom{n+k-j}{2k-j} (k-j)^{2m}
\end{equation*}
It is possible to derive more polynomial identities based on binomial relations of symmetry $\binom{n}{k} = \binom{n}{n-k}$
and an identity $\binom{n}{m} \binom{m}{k} = \binom{n}{k} \binom{n-k}{m-k}$.

So for now, let's smoothly switch our discussion to another example.
Given an odd power identity~\eqref{eq:knuth-odd-power} we can observe that odd powered polynomial can be expressed in terms of
central factorial numbers and falling factorials as follows
\begin{equation*}
    n^{2m-1} = \sum_{k=1}^{m} T(2m,2k) \fallingFactorial{n+k-1}{2k-1}
\end{equation*}
Because $\binom{n}{k} = \frac{1}{k!} \fallingFactorial{n}{k}$ where $\fallingFactorial{n}{k}$ is falling factorial.

Yet another odd power identity follows by means of~\eqref{eq:knuth-odd-power} and~\eqref{eq:cfn-identity-2}, that is
\begin{equation*}
    n^{2m-1} = \sum_{k=1}^{m} \sum_{j=0}^{k} \frac{(-1)^{k-j}}{k} \binom{n+k-1}{2k-1} \binom{2k}{k-j} j^{2m}
\end{equation*}
By means of binomial identity $\frac{k}{n} \binom{n}{k} = \binom{n-1}{k-1}$ we get
\begin{equation*}
    n^{2m-1} = \sum_{k=1}^{m} \sum_{j=0}^{k} \frac{(-1)^{k-j}}{k} \frac{2k}{n+k} \binom{n+k}{2k} \binom{2k}{k-j} j^{2m}
\end{equation*}
Collapsing common terms $k$ and applying the binomial identity
$\binom{n}{m} \binom{m}{k} = \binom{n}{k} \binom{n-k}{m-k}$ yields
\begin{equation*}
    n^{2m-1} = 2\sum_{k=1}^{m} \sum_{j=0}^{k} \frac{(-1)^{k-j}}{n+k} \binom{n+k}{k-j} \binom{n+j}{k+j} j^{2m}
\end{equation*}

As the final part of our discussion on odd power identities,
consider the polynomial identities based on odd power identity~\eqref{eq:knuth-odd-power}
and identity in central factorial numbers~\eqref{eq:cfn-identity-3}.
Replacing the term $(2k-1)!T(2n, 2k)$ in~\eqref{eq:knuth-odd-power} we get
\begin{equation*}
    n^{2m-1} = \sum_{k=1}^{m} \sum_{j=0}^{2k} \frac{(-1)^{j}}{2k} \binom{n+k-1}{2k-1} \binom{2k}{j} (k-j)^{2m}
\end{equation*}
Continuing similarly with the binomial identity $\frac{k}{n} \binom{n}{k} = \binom{n-1}{k-1}$ yields
\begin{equation*}
    n^{2m-1} = \sum_{k=1}^{m} \sum_{j=0}^{2k} \frac{(-1)^{j}}{n+k} \binom{n+k}{2k} \binom{2k}{j} (k-j)^{2m}
\end{equation*}
So that now we are free applying the binomial identity $\binom{n}{m} \binom{m}{k} = \binom{n}{k} \binom{n-k}{m-k}$
to obtain
\begin{equation*}
    n^{2m-1} = \sum_{k=1}^{m} \sum_{j=0}^{2k} \frac{(-1)^{j}}{n+k} \binom{n+k}{j} \binom{n+k-j}{2k-j} (k-j)^{2m}
\end{equation*}

The remaining opportunity to show a few more power identities lays in replacing the corresponding coefficient $T(n,k)$
in~\eqref{eq:knuth-power-identity} by identities
in central factorial numbers~\eqref{eq:cfn-identity-1},~\eqref{eq:cfn-identity-2},~\eqref{eq:cfn-identity-3}