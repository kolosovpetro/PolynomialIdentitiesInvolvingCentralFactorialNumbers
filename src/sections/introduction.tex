Central factorial numbers quietly often appear in literature, like for instance, in Riordan's
Combinatorial identities~\cite{riordan1968combinatorial}
\begin{equation*}
    T(n,k) = \frac{1}{k!} \sum_{j=0}^{k} \binom{k}{j} (-1)^{j} \left( \frac{1}{2}k - j \right)^{n}
\end{equation*}
Also, the book~\cite{carlitz_riordan_1963} references central factorial numbers as
\begin{equation*}
    K_{rs} = \frac{1}{(2s)!} \sum_{t=0}^{2s} (-1)^t \binom{2s}{t} (s-t)^{2r+2}
\end{equation*}
D. E. Knuth gives the following recurrence for the central factorial numbers~\cite{knuth1993johann}
\begin{equation*}
    T(2m+2, 2k) = k^2 T(2m,2k) + T(2m, 2k - 2)
\end{equation*}
In The On-Line Encyclopedia of Integer Sequences, central factorial numbers appear to be defined
via the following recurrence
\begin{equation*}
    \begin{cases}
        T(n,1) &=1 \\
        T(n,n) &=1 \\
        T(n,k) &=T(n-1, k-1) + k^2 T(n-1, k)
    \end{cases}
\end{equation*}
It is important to note that central factorial numbers are closely related to the central difference operator $\delta$,
Newton interpolation formula and central factorials $\centralFactorial{x}{n}$ and could be derived respectively.
The derivation of central factorial numbers by means of central difference operator $\delta$,
Newton interpolation formula and central factorials $\centralFactorial{x}{n}$ is shown at~\cite{scheuer2020mathstackexchange}.
