\begin{itemize}
    \item $T(n,k)$ recursively defines central factorial numbers of the second kind
    (in the context of Knuth and Riordan (see references)).
    It is defined in mathematica package as \texttt{CentralFactorialNumber1}
    \begin{equation*}
        \begin{cases}
            T(n,1) &= 1 \\
            T(n,n) &= 1 \\
            T(n,k) &= T(n-1, k-1) + k^2 T(n-1, k)
        \end{cases}
    \end{equation*}
    \item Identity in central factorial numbers defined as \texttt{CFNIdentity1} in mathematica package
    \begin{equation*}
    (2k-1)! T(2n,2k) = \frac{1}{k} \sum_{j=0}^{k} (-1)^j \binom{2k}{j} (k-j)^{2n}
    \end{equation*}
    \item Identity in central factorial numbers defined as \texttt{CFNIdentity2} in mathematica package
    \begin{equation*}
    (2k-1)! T(2n,2k) = \frac{1}{k} \sum_{j=0}^{k} (-1)^{k-j} \binom{2k}{k-j} j^{2n}
    \end{equation*}
    \item Identity in central factorial numbers defined as \texttt{CFNIdentity3} in mathematica package
    \begin{equation*}
    (2k-1)! T(2n, 2k) = \frac{1}{2k} \sum_{j=0}^{2k} (-1)^{j} \binom{2k}{j} (k-j)^{2n}
    \end{equation*}
\end{itemize}