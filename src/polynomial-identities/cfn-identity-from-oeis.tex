\subsection{Central factorial numbers from OEIS}\label{subsec:central-factorial-numbers-from-oeis}
$T(n,k)$ recursively defines central factorial numbers of the second kind.
Let be $U(n,k) = T(2n,2k)$ then
\begin{equation*}
    U(n,k) = \begin{cases}
                 1, & \text{if } k=1; \\
                 1, & \text{if } k=n; \\
                 U(n-1, k-1) + k^2 U(n-1, k), & \text{otherwise}
    \end{cases} \quad
    (CentralFactorialNumber1)
\end{equation*}
From OEIS, note that this is not Central factorial number itself
\begin{equation*}
    T_{\mathrm{OEIS}} (n,k) = \frac{1}{m} \sum_{j=0}^{m} (-1)^{j} \binom{2m}{j} (m-j)^{2n}
\end{equation*}
where $m=n-k+1$.
So that
\begin{equation}
    \begin{split}
        T_{\mathrm{OEIS}} (n,k) = \frac{1}{n-k+1} \sum_{j=0}^{n-k+1} (-1)^{j} \binom{2[n-k+1]}{j} ([n-k+1]-j)^{2n}
    \end{split}\label{eq:luschny-oeis}
\end{equation}
Furthermore, $T_{\mathrm{OEIS}}$ may be turned into changing the summation order from $n-k+1$ to $k$
\begin{equation*}
    T_{\mathrm{OEIS}} (n, n-k) = \frac{1}{k} \sum_{j=0}^{k} (-1)^{j} \binom{2k}{j} (k-j)^{2n}
\end{equation*}