\documentclass[12pt,letterpaper,oneside,reqno]{amsart}
\usepackage{amsfonts}
\usepackage{amsmath}
\usepackage{amssymb}
\usepackage{amsthm}
\usepackage{float}
\usepackage{mathrsfs}
\usepackage{colonequals}
\usepackage[font=small,labelfont=bf]{caption}
\usepackage[left=1in,right=1in,bottom=1in,top=1in]{geometry}
\usepackage[pdfpagelabels,hyperindex,colorlinks=true,linkcolor=blue,urlcolor=magenta,citecolor=green]{hyperref}
\usepackage{graphicx}
\linespread{1.7}
\emergencystretch=1em
\usepackage{array}
\usepackage{etoolbox}
\apptocmd{\sloppy}{\hbadness 10000\relax}{}{}
\raggedbottom

\newcommand \anglePower [2]{\langle #1 \rangle \sp{#2}}
\newcommand \bernoulli [2][B] {{#1}\sb{#2}}
\newcommand \curvePower [2]{\{#1\}\sp{#2}}
\newcommand \coeffA [3][A] {{\mathbf{#1}} \sb{#2,#3}}
\newcommand \polynomialP [4][P]{{\mathbf{#1}}\sp{#2} \sb{#3}(#4)}

% ordinary derivatives
\newcommand \derivative [2] {\frac{d}{d #2} #1}                              % 1 - function; 2 - variable;
\newcommand \pderivative [2] {\frac{\partial #1}{\partial #2}}               % 1 - function; 2 - variable;
\newcommand \qderivative [1] {D_{q} #1}                                      % 1 - function
\newcommand \nqderivative [1] {D_{n,q} #1}                                   % 1 - function
\newcommand \qpowerDerivative [1] {\mathcal{D}_q #1}                         % 1 - function;
\newcommand \finiteDifference [1] {\Delta #1}                                % 1 - function;
\newcommand \pTsDerivative [2] {\frac{\partial #1}{\Delta #2}}               % 1 - function; 2 - variable;

% high order derivatives
\newcommand \derivativeHO [3] {\frac{d^{#3}}{d {#2}^{#3}} #1}                % 1 - function; 2 - variable; 3 - order
\newcommand \pderivativeHO [3]{\frac{\partial^{#3}}{\partial {#2}^{#3}} #1}
\newcommand \qderivativeHO [2] {D_{q}^{#2} #1}                               % 1 - function; 2 - order
\newcommand \qpowerDerivativeHO [2] {\mathcal{D}_{q}^{#2} #1}                % 1 - function; 2 - order
\newcommand \finiteDifferenceHO [2] {\Delta^{#2} #1}                         % 1 - function; 2 - order
\newcommand \pTsDerivativeHO [3] {\frac{\partial^{#3}}{\Delta {#2}^{#3}} #1} % 1 - function; 2 - variable;
\newcommand \centralFactorial [2] {#1^{[#2]}}
\newcommand \fallingFactorial [2] {\left(#1 \right)^{\underline{#2}}}
\newcommand{\stirlingii}{\genfrac{\{}{\}}{0pt}{}}
\newcommand{\eulerianNumber}{\genfrac{\langle}{\rangle}{0pt}{}}

\newtheorem{thm}{Theorem}[section]
\newtheorem{cor}[thm]{Corollary}
\newtheorem{lem}[thm]{Lemma}
\newtheorem{examp}[thm]{Example}
\newtheorem{conj}[thm]{Conjecture}
\newtheorem{defn}[thm]{Definition}

\numberwithin{equation}{section}

\title[Polynomial identities auxiliary]
{Polynomial identities auxiliary}
\author[Petro Kolosov]{Petro Kolosov}
\keywords{
    Polynomials,
    Polynomial identities,
    Faulhaber's formula,
    Cental Factorial Numbers
}
\subjclass[2010]{26E70, 05A30}
\date{\today}
\hypersetup{
    pdftitle={Polynomial identities auxiliary},
    pdfsubject={
        Polynomials,
        Polynomial identities,
        Faulhaber's formula,
        Cental Factorial Numbers
    },
    pdfauthor={Petro Kolosov},
    pdfkeywords={
        Polynomials,
        Polynomial identities,
        Faulhaber's formula,
        Cental Factorial Numbers
    }
}
\begin{document}
    \begin{abstract}
        Polynomial identities auxiliary
    \end{abstract}

    \maketitle

    \tableofcontents


    \section{Polynomial identities auxiliary}\label{sec:polynomial-identities-auxiliary}
    \subsection{Central factorial numbers}\label{subsec:central-factorial-numbers}

\begin{equation*}
(2k-1)
    !T(2n,2k) = \frac{1}{k} \sum_{j=0}^{k} (-1)^j \binom{2k}{j} (k-j)^{2n} \quad
    (CFNIdentity1)
\end{equation*}
\begin{equation*}
(2k-1)
    !T(2n,2k) = \frac{1}{k} \sum_{j=0}^{k} (-1)^{k-j} \binom{2k}{k-j} j^{2n} \quad
    (CFNIdentity2)
\end{equation*}
\begin{equation*}
(2k-1)
    !T(2n, 2k) = \frac{1}{2k} \sum_{j=0}^{2k} \binom{2k}{j} (-1)^{j} (k-j)^{2n} \quad
    (CFNIdentity3)
\end{equation*}
\begin{equation*}
    T(n,k) = \frac{1}{k!} \sum_{j=0} \binom{k}{j} (-1)^{j} \left( \frac{1}{2}k - j \right)^{n} \quad
    (CentralFactorialNumber2)
\end{equation*}
\begin{equation*}
    T(2n,2k) = 2 \sum_{j=1}^{k} (-1)^{k-j} \frac{j^{2n}}{(k-j)! (k+j)!} \quad
    (CentralFactorialNumber3)
\end{equation*}
Hypothesis
\begin{equation*}
    T(n,k) = \sum_{j=1}^{2k} (-1)^{k-j} \frac{j^{n}}{(k-j)! (k+j)!}
\end{equation*}
    \subsection{Central factorial numbers from OEIS}\label{subsec:central-factorial-numbers-from-oeis}
$T(n,k)$ recursively defines central factorial numbers of the second kind.
Let be $U(n,k) = T(2n,2k)$ then
\begin{equation*}
    U(n,k) = \begin{cases}
                 1, & \text{if } k=1; \\
                 1, & \text{if } k=n; \\
                 U(n-1, k-1) + k^2 U(n-1, k), & \text{otherwise}
    \end{cases} \quad
    (CentralFactorialNumber1)
\end{equation*}
From OEIS, note that this is not Central factorial number itself
\begin{equation*}
    T_{\mathrm{OEIS}} (n,k) = \frac{1}{m} \sum_{j=0}^{m} (-1)^{j} \binom{2m}{j} (m-j)^{2n}
\end{equation*}
where $m=n-k+1$.
So that
\begin{equation}
    \begin{split}
        T_{\mathrm{OEIS}} (n,k) = \frac{1}{n-k+1} \sum_{j=0}^{n-k+1} (-1)^{j} \binom{2[n-k+1]}{j} ([n-k+1]-j)^{2n}
    \end{split}\label{eq:luschny-oeis}
\end{equation}
Furthermore, $T_{\mathrm{OEIS}}$ may be turned into changing the summation order from $n-k+1$ to $k$
\begin{equation*}
    T_{\mathrm{OEIS}} (n, n-k) = \frac{1}{k} \sum_{j=0}^{k} (-1)^{j} \binom{2k}{j} (k-j)^{2n}
\end{equation*}
%    From OEIS, recurrence relation
$T(2n, 2k)$ defines central factorial numbers of the second kind
\begin{equation*}
    \begin{cases}
        T(n,1) &=1 \\
        T(n,n) &=1 \\
        T(n,k) &=T(n-1, k-1) + k^2 T(n-1, k)
    \end{cases} \quad
    (CentralFactorialNumber1)
\end{equation*}
From OEIS, note that this is not Central factorial number itself, this formula is in the mathematica package as
\texttt{OEISFormula}
\begin{equation*}
    T_{\mathrm{OEIS}} (n,k) = \frac{1}{m} \sum_{j=0}^{m} (-1)^{j} \binom{2m}{j} (m-j)^{2n}
\end{equation*}
where $m=n-k+1$.
So that
\begin{equation}
    \begin{split}
        T_{\mathrm{OEIS}} (n,k) = \frac{1}{n-k+1} \sum_{j=0}^{n-k+1} (-1)^{j} \binom{2[n-k+1]}{j} ([n-k+1]-j)^{2n}
    \end{split}\label{eq:luschny-oeis}
\end{equation}
Furthermore, $T_{\mathrm{OEIS}}$ may be turned into changing the summation order from $n-k+1$ to $k$
\begin{equation*}
    T_{\mathrm{OEIS}} (n, n-k) = \frac{1}{k} \sum_{j=0}^{k} (-1)^{j} \binom{2k}{j} (k-j)^{2n}
\end{equation*}
    \subsection{Knuth's formula for odd power: approach 1}\label{subsec:knuth's-formula-for-odd-power-approach-1}
\begin{equation*}
    n^{2m-1} = \sum_{k=1}^{m} (2k-1)! T(2m,2k) \binom{n+k-1}{2k-1} \quad
    (OddPowerIdentity(1,2,3))
\end{equation*}
\begin{equation*}
    n^{2m-1} = \sum_{k=1}^{m} T(2m,2k) \fallingFactorial{n+k-1}{2k-1} \quad
    (OddPowerIdentity4)
\end{equation*}
Substituting $(2k-1)!T(2n,2k) = \frac{1}{k} \sum_{j=0}^{k} (-1)^j \binom{2k}{j} (k-j)^{2n}$ we get
\begin{equation*}
    n^{2m-1} = \sum_{k=1}^{m} \sum_{j=0}^{k} \frac{(-1)^j}{k} \binom{2k}{j} \binom{n+k-1}{2k-1} (k-j)^{2m} \quad
    (OddPowerIdentity11)
\end{equation*}
By means of binomial identity $\frac{k}{n} \binom{n}{k} = \binom{n-1}{k-1}$
\begin{equation*}
    n^{2m-1} = \sum_{k=1}^{m} \sum_{j=0}^{k} \frac{(-1)^{j}}{k} \frac{2k}{n+k} \binom{n+k}{2k} \binom{2k}{j} (k-j)^{2m} \quad
    (OddPowerIdentity12)
\end{equation*}
Collapsing common terms and by means of binomial identity $\binom{n}{m} \binom{m}{k} = \binom{n}{k} \binom{n-k}{m-k}$ we get
\begin{equation*}
    n^{2m-1} = 2\sum_{k=1}^{m} \sum_{j=0}^{k} \frac{(-1)^{j}}{n+k} \binom{n+k}{j} \binom{n+k-j}{2k-j} (k-j)^{2m} \quad
    (OddPowerIdentity13)
\end{equation*}
Because the symmetry of binomial coefficients $\binom{n+k}{k-j} = \binom{n+k}{n+k-(k-j)}$ holds, we get
\begin{equation*}
    n^{2m-1} = 2\sum_{k=1}^{m} \sum_{j=0}^{k} \frac{(-1)^{j}}{n+k} \binom{n+k}{n+k-j} \binom{n+k-j}{2k-j} (k-j)^{2m} \quad
    (OddPowerIdentity14)
\end{equation*}
By means of binomial identity $\binom{n}{m} \binom{m}{k} = \binom{n}{k} \binom{n-k}{m-k}$ we get
\begin{equation*}
    n^{2m-1} = 2\sum_{k=1}^{m} \sum_{j=0}^{k} \frac{(-1)^{j}}{n+k} \binom{n+k}{2k-j} \binom{n-k+j}{n-k} (k-j)^{2m} \quad
    (OddPowerIdentity15)
\end{equation*}
    \subsection{Knuth's formula for odd power: approach 2}\label{subsec:knuth's-formula-for-odd-power-approach-2}
\begin{equation}
    n^{2m-1} = \sum_{k=1}^{m} (2k-1)! T(2m,2k) \binom{n+k-1}{2k-1}\label{eq:knuth-general}
\end{equation}
Equation~\eqref{eq:knuth-general} is validated via Mathematica functions:
\textit{OddPowerIdentity1}, \textit{OddPowerIdentity2}, \textit{OddPowerIdentity3}.
Substituting $(2k-1)! T(2n,2k) = \frac{1}{k} \sum_{j=0}^{k} (-1)^{k-j} \binom{2k}{k-j} j^{2n}$ we get
\begin{equation*}
    n^{2m-1} = \sum_{k=1}^{m} \sum_{j=0}^{k} \frac{(-1)^{k-j}}{k} \binom{n+k-1}{2k-1} \binom{2k}{k-j} j^{2m} \quad
    (OddPowerIdentity21)
\end{equation*}
By means of binomial identity $\frac{k}{n} \binom{n}{k} = \binom{n-1}{k-1}$
\begin{equation*}
    n^{2m-1} = \sum_{k=1}^{m} \sum_{j=0}^{k} \frac{(-1)^{k-j}}{k} \frac{2k}{n+k} \binom{n+k}{2k} \binom{2k}{k-j} j^{2m} \quad
    (OddPowerIdentity22)
\end{equation*}
Collapsing common terms we get
\begin{equation*}
    n^{2m-1} = 2\sum_{k=1}^{m} \sum_{j=0}^{k} \frac{(-1)^{k-j}}{n+k} \binom{n+k}{2k} \binom{2k}{k-j} j^{2m} \quad
    (OddPowerIdentity23)
\end{equation*}
By means of binomial identity $\binom{n}{m} \binom{m}{k} = \binom{n}{k} \binom{n-k}{m-k}$ we get
\begin{equation*}
    n^{2m-1} = 2\sum_{k=1}^{m} \sum_{j=0}^{k} \frac{(-1)^{k-j}}{n+k} \binom{n+k}{k-j} \binom{n+j}{k+j} j^{2m} \quad
    (OddPowerIdentity24)
\end{equation*}
Because the symmetry of binomial coefficients $\binom{n+k}{k-j} = \binom{n+k}{n+k-(k-j)}$ holds, we get
\begin{equation*}
    n^{2m-1} = 2\sum_{k=1}^{m} \sum_{j=0}^{k} \frac{(-1)^{k-j}}{n+k} \binom{n+k}{n+j} \binom{n+j}{k+j} j^{2m} \quad
    (OddPowerIdentity25)
\end{equation*}
By means of binomial identity $\binom{n}{m} \binom{m}{k} = \binom{n}{k} \binom{n-k}{m-k}$ we get
\begin{equation*}
    n^{2m-1} = 2\sum_{k=1}^{m} \sum_{j=0}^{k} \frac{(-1)^{k-j}}{n+k} \binom{n+k}{k+j} \binom{n-j}{n-k} j^{2m} \quad
    (OddPowerIdentity26)
\end{equation*}
    \subsection{Knuth's formula for odd power: approach 3}\label{subsec:knuth's-formula-for-odd-power-approach-3}
\begin{equation*}
    n^{2m-1} = \sum_{k=1}^{m} (2k-1)! T(2m,2k) \binom{n+k-1}{2k-1} \quad
    (OddPowerIdentity(1,2,3))
\end{equation*}
\begin{equation*}
    n^{2m-1} = \sum_{k=1}^{m} T(2m,2k) \fallingFactorial{n+k-1}{2k-1} \quad
    (OddPowerIdentity4)
\end{equation*}
Let be
\begin{equation*}
(2k-1)
    !T(2n, 2k) = \frac{1}{2k} \sum_{j=0}^{2k} \binom{2k}{j} (-1)^{j} (k-j)^{2n}
\end{equation*}
And
\begin{equation*}
    n^{2m-1} = \sum_{k=1}^{m} (2k-1)! T(2m,2k) \binom{n+k-1}{2k-1}
\end{equation*}
Then
\begin{equation*}
    n^{2m-1} = \sum_{k=1}^{m} \frac{1}{2k} \sum_{j=0}^{2k} \binom{2k}{j} (-1)^{j} (k-j)^{2m} \binom{n+k-1}{2k-1}
\end{equation*}
\begin{equation*}
    n^{2m-1} = \sum_{k=1}^{m} \sum_{j=0}^{2k} \frac{(-1)^{j}}{2k} \binom{n+k-1}{2k-1} \binom{2k}{j} (k-j)^{2m} \quad
    (OddPowerIdentity31)
\end{equation*}
By means of binomial identity $\frac{k}{n} \binom{n}{k} = \binom{n-1}{k-1}$
\begin{equation*}
    n^{2m-1} = \sum_{k=1}^{m} \sum_{j=0}^{2k} \frac{(-1)^{j}}{2k} \frac{2k}{n+k} \binom{n+k}{2k} \binom{2k}{j} (k-j)^{2m}
\end{equation*}
Collapsing common terms we get
\begin{equation*}
    n^{2m-1} = \sum_{k=1}^{m} \sum_{j=0}^{2k} \frac{(-1)^{j}}{n+k} \binom{n+k}{2k} \binom{2k}{j} (k-j)^{2m} \quad
    (OddPowerIdentity32)
\end{equation*}
By means of binomial identity $\binom{n}{m} \binom{m}{k} = \binom{n}{k} \binom{n-k}{m-k}$ we get
\begin{equation*}
    n^{2m-1} = \sum_{k=1}^{m} \sum_{j=0}^{2k} \frac{(-1)^{j}}{n+k} \binom{n+k}{j} \binom{n+k-j}{2k-j} (k-j)^{2m} \quad
    (OddPowerIdentity33)
\end{equation*}
Because the symmetry of binomial coefficients $\binom{n+k}{j} = \binom{n+k}{n+k-j}$ holds, we get
\begin{equation*}
    n^{2m-1} = \sum_{k=1}^{m} \sum_{j=0}^{2k} \frac{(-1)^{j}}{n+k} \binom{n+k}{n+k-j} \binom{n+k-j}{2k-j} (k-j)^{2m}
\end{equation*}
By means of binomial identity $\binom{n}{m} \binom{m}{k} = \binom{n}{k} \binom{n-k}{m-k}$ we get
\begin{equation*}
    n^{2m-1} = \sum_{k=1}^{m} \sum_{j=0}^{2k} \frac{(-1)^{j}}{n+k} \binom{n+k}{2k-j} \binom{n-k+j}{n-k} (k-j)^{2m} \quad
    (OddPowerIdentity34)
\end{equation*}
    \subsection{Central factorials power identity}\label{subsec:central-factorials-power-identity}
Central factorials
\begin{equation*}
    \centralFactorial{x}{n} = x \fallingFactorial{x+\frac{n}{2}-1}{n-1} \quad
    (CentralFactorial1)
\end{equation*}
\begin{equation*}
    \centralFactorial{x}{n} = x \left( x+\frac{n}{2}-1 \right) \left( x+\frac{n}{2}-1 \right) \cdots \left (x+\frac{n}{2}-1 \right)
\end{equation*}
\begin{equation*}
    \centralFactorial{x}{n} = x \prod_{k=1}^{n-1} \left( x+\frac{n}{2}-k \right)
    (CentralFactorial2)
\end{equation*}
Then we have power identity given by Knuth
\begin{equation*}
    x^m = \sum_{k=1}^{m} T(m, k) \centralFactorial{x}{k} \quad
    (PowerIdentity1, PowerIdentity2)
\end{equation*}

\end{document}