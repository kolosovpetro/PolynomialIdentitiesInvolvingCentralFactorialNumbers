\documentclass[12pt,letterpaper,oneside,reqno]{amsart}
\usepackage{amsfonts}
\usepackage{amsmath}
\usepackage{amssymb}
\usepackage{amsthm}
\usepackage{float}
\usepackage{mathrsfs}
\usepackage{colonequals}
\usepackage[font=small,labelfont=bf]{caption}
\usepackage[left=1in,right=1in,bottom=1in,top=1in]{geometry}
\usepackage[pdfpagelabels,hyperindex,colorlinks=true,linkcolor=blue,urlcolor=magenta,citecolor=green]{hyperref}
\usepackage{graphicx}
\linespread{1.7}
\emergencystretch=1em
\usepackage{array}
\usepackage{etoolbox}
\apptocmd{\sloppy}{\hbadness 10000\relax}{}{}
\raggedbottom

\newcommand \anglePower [2]{\langle #1 \rangle \sp{#2}}
\newcommand \bernoulli [2][B] {{#1}\sb{#2}}
\newcommand \curvePower [2]{\{#1\}\sp{#2}}
\newcommand \coeffA [3][A] {{\mathbf{#1}} \sb{#2,#3}}
\newcommand \polynomialP [4][P]{{\mathbf{#1}}\sp{#2} \sb{#3}(#4)}

% ordinary derivatives
\newcommand \derivative [2] {\frac{d}{d #2} #1}                              % 1 - function; 2 - variable;
\newcommand \pderivative [2] {\frac{\partial #1}{\partial #2}}               % 1 - function; 2 - variable;
\newcommand \qderivative [1] {D_{q} #1}                                      % 1 - function
\newcommand \nqderivative [1] {D_{n,q} #1}                                   % 1 - function
\newcommand \qpowerDerivative [1] {\mathcal{D}_q #1}                         % 1 - function;
\newcommand \finiteDifference [1] {\Delta #1}                                % 1 - function;
\newcommand \pTsDerivative [2] {\frac{\partial #1}{\Delta #2}}               % 1 - function; 2 - variable;

% high order derivatives
\newcommand \derivativeHO [3] {\frac{d^{#3}}{d {#2}^{#3}} #1}                % 1 - function; 2 - variable; 3 - order
\newcommand \pderivativeHO [3]{\frac{\partial^{#3}}{\partial {#2}^{#3}} #1}
\newcommand \qderivativeHO [2] {D_{q}^{#2} #1}                               % 1 - function; 2 - order
\newcommand \qpowerDerivativeHO [2] {\mathcal{D}_{q}^{#2} #1}                % 1 - function; 2 - order
\newcommand \finiteDifferenceHO [2] {\Delta^{#2} #1}                         % 1 - function; 2 - order
\newcommand \pTsDerivativeHO [3] {\frac{\partial^{#3}}{\Delta {#2}^{#3}} #1} % 1 - function; 2 - variable;

\newtheorem{thm}{Theorem}[section]
\newtheorem{cor}[thm]{Corollary}
\newtheorem{lem}[thm]{Lemma}
\newtheorem{examp}[thm]{Example}
\newtheorem{conj}[thm]{Conjecture}
\newtheorem{defn}[thm]{Definition}

\numberwithin{equation}{section}

\title[Binomial identities]
{Binomial identities}
\author[Petro Kolosov]{Petro Kolosov}
\email{kolosovp94@gmail.com}
\keywords{
    Polynomials,
    Polynomial identities,
    Faulhaber's formula,
    Cental Factorial Numbers
}
\urladdr{https://kolosovpetro.github.io}
\subjclass[2010]{26E70, 05A30}
\date{\today}
\hypersetup{
    pdftitle={Binomial identities},
    pdfsubject={
        Polynomials,
        Polynomial identities,
        Faulhaber's formula,
        Cental Factorial Numbers
    },
    pdfauthor={Petro Kolosov},
    pdfkeywords={
        Polynomials,
        Polynomial identities,
        Faulhaber's formula,
        Cental Factorial Numbers
    }
}
\begin{document}
    \begin{abstract}
        Binomial identities
    \end{abstract}

    \maketitle

    \tableofcontents


    \section{Binomial identities}\label{sec:binomial-identities}
    %! suppress = MissingLabel

\subsection{Part 1}\label{subsec:part-1}
\begin{equation}
    \binom{n}{k} = \binom{n-1}{k} + \binom{n-1}{k-1}
\end{equation}
\begin{equation}
    \binom{n}{k} = \frac{n^{\underline{k}}}{k!}
\end{equation}
\begin{equation}
    \sum_{r=0}^{n} \binom{r}{c} = \binom{n+1}{c+1}
\end{equation}
\begin{equation}
    \sum_{k=0}^{n} \binom{r+k}{k} = \binom{r+n+1}{n}
\end{equation}
\begin{equation}
    \sum_{k=0}^{m} \binom{n-k}{m-k} = \binom{n+1}{m}
\end{equation}
\begin{equation}
    \sum_{k=0}^{n} \binom{n-k}{k} = f_{n+1}
\end{equation}
\begin{equation}
    k \binom{n}{k} = n \binom{n-1}{k-1}
\end{equation}
\begin{equation}
    \binom{n}{m} \binom{m}{k} = \binom{n}{k} \binom{n-k}{m-k}
\end{equation}
\begin{equation}
    \sum_{j=0}^{n} \binom{n}{j} \binom{m}{k-j} = \binom{n+m}{k}
\end{equation}

\subsection{Part2}\label{subsec:part2}
\begin{equation}
    k \binom{n}{k} = n \binom{n-1}{k-1}
\end{equation}
\begin{equation}
    \frac{k}{n} \binom{n}{k} = \binom{n-1}{k-1}
\end{equation}
\begin{equation}
    \frac{k+1}{n+1} \binom{n+1}{k+1} = \binom{n}{k}
\end{equation}
\begin{equation}
    \binom{n+1}{k+1} = \frac{n+1}{k+1} \binom{n}{k}
\end{equation}

\subsection{Part 3}\label{subsec:part-3}
\begin{equation}
    \binom{t}{r} \binom{r}{k} = \binom{t}{k} \binom{t-k}{r-k}
\end{equation}
\begin{equation}
    \binom{t}{k} \binom{t-k}{r-k} = \binom{t}{t-k} \binom{t-k}{r-k} = \binom{t}{r-k} \binom{t-r+k}{t-r}
\end{equation}
\begin{equation}
    \binom{t}{k} \binom{t-k}{r-k} = \binom{t}{k} \binom{t-k}{t-r}
\end{equation}
\begin{equation}
    \binom{t}{k} \binom{t-k}{t-r} = \binom{t}{t-k} \binom{t-k}{t-r}
\end{equation}
\subsection{Part 4}\label{subsec:part-4}
\begin{equation}
    \binom{t}{r} \binom{r}{k} = \binom{t}{k} \binom{t-k}{r-k} = \binom{t}{t-k} \binom{t-k}{r-k} = \binom{t}{r-k} \binom{t-r+k}{t-r} = \binom{t}{r-k} \binom{t-r+k}{k}
\end{equation}

\end{document}